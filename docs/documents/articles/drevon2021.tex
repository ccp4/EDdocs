%%%%necessary for acta crystallographica

\section{Introduction}

\subsection{Motivation}
Since the first experimental demonstration of electron diffraction in 1928,
theory of dynamical diffraction has been developed~\cite{Bethe1928,CowleyBook}.
Besides, multiple scattering~\cite{Korringa1947,Kohn1954,Korringa1994,Dederichs1971}
has also extensively been studied in solid state physics over the 20th century.
In fact, even at very high electron energies commonly used in modern transmission electron
microscopes, the electron-atom interaction is very strong and the the kinematic
theory of diffraction is not theoretically valid for crystals thicker than a
few nm\cite{GlaeserDowning1993,SubramanianSpence2015}. In practice, crystal
growth cannot be controlled to such a degree of accuracy and nano-crystals
of organic compounds are usually on the order of tens to hundreads of nanometers.
This is known as a challenging aspect of high energy electron diffraction (HEED) as it
should significantly complicate the structure determination process.
However, successfull structure determination based on the standard kinematic
theory used in X-ray diffraction have regularly been demonstrated over the
past 10 years\cite{Nannenga2014,Nannenga2019} suggesting that dynamical diffraction may
not affect the diffraction intensities as much as the theory suggests.
Although, dynamical refinement based structure determination~\cite{Palatinus2013} usually
leads to better intensity predictions~\cite{Gemmi2019b}, the agreement
between theory and experiment is still significantly worse than those obtained
for X-rays~\cite{Oleynikov2007}. It is therefore crucial to develop more
accurate models of electron diffraction by crystals.

\subsection{state of the art}
The multislice(MS)~\cite{CowleyMoodie1957} and Blochwave(BW)~\cite{Bethe1928}
approaches are the most popular methods for simulating scattering of high energy
electrons in crystals.
The MS is particularly well suited for solving large structures as it involves
successive convolutions which can be very efficiently computed with the Fast
Fourier Transform(FFT)~\cite{Ishizuka1977}. To avoid aliasing transverse periodic
boundary conditions must be met which is only possible for orthorombic
structures in zone axis orientations. Although small beam tilt can also be
used~\cite{Ishizuka1982,ChenDyck1997}, simluations with arbitrary orientations
must be performed by simulating a full crystal with added zero padding. This
can quickly become computationnally challenging.

On the other hand, BW method can simulate small structures in any arbitrary
orientations. Although some efficient implementation~\cite{Zuo1995} can simulate
moderately large structures, BW cannot be applied to large structures due to
the unfavorable scaling behaviour of the matrix diagonalization involved.
Non periodic structures, defects and solvent scattering cannot be modelled either.

Both approach can handle coherent inelastic scattering through the use of
a imaginary part of the potential. This however does not account for the
loss of coherency induced by inelastic scattering which may have dramatic
impact on the dynamical diffraction~\cite{Latychevskaia2019}.


\subsection{Contribution and outline}

The purpose of this paper is to propose an alternative real space approach
to the scattering of fast electrons by light-atom structures based on
the T-matrix formalism.
The T-matrix has been extensively applied to various areas of physics including
electromagnetics~\cite{Hamid1990,Hamid1990_b,Eremin1995}, optics~\cite{Moine2005}
and acoustics~\cite{Silva2012,Godin2011}.
Although not computationnally competitive with multislice
for large systems, this approach provides an exact solution to Schr{"\o}dinger's
equation for an assemble of spherically symmetric effective atomic potentials in
the independent atom model(IAM).
An intuitive understanding of multiple scattering in the forward scattering
approximation is presented and compared with existing intepretations.
The validity of the forward scattering approximation and the phase grating
approximation used by multislice are then discussed.
Conclusions are drawn and extensions of this approach to account for incoherent
inelastic scattering are outlined.



%%%%%%%%%%%%%%%%%%%%%%%%%%%%%%%%%%%%%%%%%%%%%%%%%%%%%%%%%%%%%%%%%%%%%%%%%
%%%% Theory
%%%%%%%%%%%%%%%%%%%%%%%%%%%%%%%%%%%%%%%%%%%%%%%%%%%%%%%%%%%%%%%%%%%%%%%%%
\section{Theory}

The problem of the scattering of fast electrons by an assembly of atoms is
found by solving Schr\"{o}odinger's equation :
%
\begin{equation}\label{eq:schro}
  \left[-\frac{\hbar^2}{2m_e}\grad_{\bb r}^2 - eV(\bb r) \right] \Psi = E\Psi
\end{equation}
%
where $\hbar$ is plank's constant, $m_e$ the mass of the electron,
$e$ the elementary charge, $V(\bb r)$ is the spatially varying electrostatic
potential created by the atoms and $E$ the energy of the incident electrons.
The wave function is sought as a sum of an incident wave $\Psi^{(i)}$
and a scattered wave $\Psi^{(s)}$.

\subsection{T-matrix formulation}

\begin{figure}
  \centering
  \includegraphics[height=0.3\textheight]{figures/Tmatrix.png}
\caption{The problem solved by the T-matrix}\label{fig:Tmatrix}
\end{figure}

In its standard form, the T-matrix solves for the case where the electrostatic
potential is uniform constant inside non overlapping spheres and the
incident wave is described by a plane wave of wavenumber $k_0$.
The setup shown in figure~\ref{fig:Tmatrix}.
The formulation is well established and the theory is outlined for the purpose
of introducing the forward multiple scattering approximation picture.

in each domain are given by the solutions to the Helmholtz equation
in spherical coordinates :
%
\begin{eqnarray*}
  \grad_{\bb r_p}^2\Psi + k_p^2 \Psi &=& 0\\
  k_p &=& k_0\sqrt{1+\frac{V_p}{E}}\\
  k_0 &=& \sqrt{\frac{2m_eE}{\hbar^2}}\\
\end{eqnarray*}
%
where $V_p\ge 0$ is the constant attractive potential inside sphere
$p$ of radius $a_p$ centered at $\bb d_p$.
The scattered wave function inside and outside of sphere $p$ can
then be expressed as :
%
\begin{eqnarray*}
  \Psi_p^{(in )}(\bb r_p) &=& \sum_{l=0}^{\infty}j_l(k_pr_p)\sum_{m=-l}^{m=l}a_{p;lm}Y_l^m(\theta_p,\phi_p)\\
  \Psi_p^{(out)}(\bb r_p) &=& \sum_{l=0}^{\infty}\hl(k_0r_p)\sum_{m=-l}^{m=l}b_{p;lm}Y_l^m(\theta_p,\phi_p)\\
\end{eqnarray*}
%
where $p=1..N$, $E$, $k_0=2\pi/\lambda$ are the energy and wave number of the incident wave,
$k_p$, $V_p$ the wave number and constant potential inside the sphere,
$j_l$ and $\hl$ are the spherical Bessel and Hankel functions of the first
kind, $Y_l^m$ are the spherical harmonics or order $l$ and azimuthal order $m$.
Note that these equations are expressed in the reference frame of each
sphere $p$ hence the use of variable $\bb r_p$.

The unknown coefficients $a_{p;lm}$, $b_{p;lm}$ are found by imposing the
continuity of the wave function and its radial derivative at the surface of each
sphere $p$ :
%
\begin{eqnarray*}
      \Big(\sum_{q=1}^N \Psi_q^{(out)}+\Psi^{(i)}\Big)\big|_{r_p=a_p}
  &=& \Big(\Psi_p^{(in)}\Big)\big|_{r_p=a_p} \\
      \dP_{r_p}\Big(\sum_{q=1}^N \Psi_q^{(out)}+\Psi^{(i)}\Big)\big|_{r_p=a_p}
  &=& \dP_{r_p}\Big(\Psi_p^{(in)}\Big)\big|_{r_p=a_p} \\
\end{eqnarray*}
%
where $f^{(i)}$ is the incident electron wavefunction and $a_p$ the
radius of sphere $p$.

Using the orthogonality of the spherical harmonics
% $\int_{\Omega} Y_{l}^{m}Y_{l'}^{m'*}d\Omega=\delta_{l,l'}\delta_{m,m'}$,
the following linear system yields the unknown coefficients :
%
\begin{eqnarray}
  a_{p;lm} &=& u_{p;l}c_{lm} + u_{p;l}\sum_{q\neq p}^{N} \nonumber\\
    &&~~\sum_{\nu=0}^{\infty}\sum_{\mu=-\nu}^{\mu=\nu}
      a_{\nu,\mu;l,m}^{(out-in)}(\bb d_{pq})b_{q;\nu\mu}\label{eq:aplm}\\
      % a_{\nu,\mu;l,m}^{(out-in)}(d_{pq},\theta_{pq},\phi_{pq})b_{q;\nu\mu}\\
  b_{p;lm} &=& v_{p;l}c_{lm} + v_{p;l} \sum_{q\neq p}^{N}\nonumber\\
    &&~~\sum_{\nu=0}^{\infty}\sum_{\mu=-\nu}^{\mu=\nu}
      a_{\nu,\mu;l,m}^{(out-in)}(\bb d_{pq})b_{q;\nu\mu}\label{eq:bplm}\\\nonumber
\end{eqnarray}
%
where the translational addition theorem\cite{Dufva2008} has been
used to express the field scattered by sphere $q$ in the reference
frame of sphere $p$ that is $f_q^{(out)}(\bb r_p)$.
This operation involves the coupling coefficients
% $a_{\nu,\mu;l,m}^{(out-in)}(d_{pq},\theta_{pq},\phi_{pq})$
$a_{\nu,\mu;l,m}^{(out-in)}(\bb d_{pq})$
where $\bb d_{pq}=\bb d_q-\bb d_p$.
%
% \begin{eqnarray*}
%   d_{pq}  &=& ||\bb d_q-\bb d_p|| \\
%   \theta_{pq} &=& \arccos\left(\frac{z_q-z_p}{d_{pq} }\right)\\
%   \phi_{pq}   &=& \arctan\left(\frac{y_q-y_p}{x_q-x_p}\right)\\
% \end{eqnarray*}
%

The coefficient $c_{lm}$ are related to the incident wave.
In the case of a plane wave $e^{j\bb k_0\cdot\bb r}$, $j=\sqrt{-1}$,
the addition theorem is used to expand the plane wave on
the family basis os Spherical Bessel soluions :
\begin{eqnarray*}
  c_{lm}  &=& 4\pi j^l Y_l^{m*}\left(\alpha_i,\phi_a\right)e^{jk_0d_p\zeta_p}\\
  \zeta_p     &=& \sin(\Theta_p)\sin(\Phi_p)\sin(\alpha_i)+
                \cos(\Theta_p)\cos(\alpha_i)\\
\end{eqnarray*}
%
where $d_p,\Theta_p,\Phi_p$ being the spherical coordinates of the
centre of sphere $p$ in the global coordinate system,
$0\le\alpha_i\le\pi$ is the angle of incidence with respect to
the $\bb e_z$ axis, $\phi_i=\pi/2$ since the propagation is in the $(y,z)$
plane and $k_0d_p\zeta_p$ is the phase factor at sphere $p$.
The different notations as illustrated on figure~\ref{fig:Tmatrix}.

The coefficients $u_{p;l}$ and $v_{p;l}$ are expressed as :
\begin{eqnarray*}
  u_{p;l} &=& \frac{h_l^{'}(k_0a_p)j_l(k_0a_p) - h_l(k_0a_p)j_l(k_0a_p)^{'}}
    {j_l(k_pa_p)h_l^{'}(k_0a_p)-n_pj_l^{'}(k_pad_p)h_l(k_0a_p)} \\
  v_{p;l} &=& \frac{n_pj_l^{'}(k_pa_p)j_l(k_0a_p) - j_l(k_pa_p)j_l^{'}(k_0a_p)}
    {j_l(k_pa_p)h_l^{'}(k_0a_p)-n_pj_l^{'}(k_pa_p)h_l(k_0a_p)} \\
\end{eqnarray*}
%
where $z_l^{'}=\dP_{\rho}z_l(\rho)$.

Equations~(\ref{eq:aplm},\ref{eq:bplm}) can be written in a matrix notation :
%
\begin{equation}\label{eq:Tsystem}
  \big(\bb I - \bb T \big)\bb A = \bb L
\end{equation}
%
where
$\bb I$ is the identity matrix,
$\bb A$ is the unknown vector coefficients,
$\bb L$ the incident wave right hand side and
$\bb T$ is the cross-coupling matrix.


\subsection{Far field and scattering cross section}
In electron crystallography, the diffraction pattern of particular
interest which is recorded in the far field. Using the asymptotic
behaviour $\hl(k_0r_p)\approx (-j)^{l+1}\frac{e^{jk_0rp}}{k_0r_p}$
and since $\theta_p=\theta$,$\phi_p=\phi$
the far field scattering amplitude $f_p(\theta,\phi)$ from sphere $p$ can
be written as :
%
\begin{equation}\label{eq:fp_theta}
  f_p(\theta,\phi) = \sum_{l=0}^{\infty}\sum_{m=-l}^{l} (-j)^{l+1}b_{p;lm}Y_l^m(\theta,\phi)
\end{equation}
%
The total scattering amplitude is the sum of the contribution from all
individual spheres.
Since in the far field, $r_p\approx r-d_p\cos(\theta-\Theta_p)$ :
%
\begin{equation}\label{eq:f_theta}
  f(\theta,\phi) = \sum_{p=1}^{N} f_p(\theta,\phi)e^{-jk_0d_p\cos(\theta-\Theta_p)}
\end{equation}
%

The normalized differential scattering cross section is defined as :
%
\begin{equation}\label{eq:sigma}
  \frac{\sigma(\theta,\phi)}{\pi a_p^2}
  = \frac{4\pi r^2}{\pi a_p^2}
    \norm{\frac{\Psi^{(s)}(r,\theta,\phi)|}{\Psi^{(i)}(r,\theta,\phi)}}^2
  = \frac{4\bigl|f(\theta,\phi)\bigr|^2}{\left(k_0a_p\right)^2}
\end{equation}
%
where we have used
$\Psi^{(s)}(r,\theta,\phi) \underset{r\rightarrow\infty}\approx \frac{e^{jk_0r}}{k_0r}f(\theta,\phi)$.





%%%%%%%%%%%%%%%%%%%%%%%%%%%%%%%%%%%%%%%%%%%%%%%%%%%%%%%%%%%%%%%%%%%%%%%%%
%%%% Forward Multiple Scattering
%%%%%%%%%%%%%%%%%%%%%%%%%%%%%%%%%%%%%%%%%%%%%%%%%%%%%%%%%%%%%%%%%%%%%%%%%
\section{Forward scattering and multiple scattering approximations}
%
Expression~\eqref{eq:Tsystem} is a convenient way to write the system as it
readily identifies $\bb L$ as the solution to the uncoupled problem.
Indeed,
$a_{p;lm}=c_{lm}u_{p;l}$,
$b_{p;lm}=c_{lm}v_{p;l}$
are them well known analytical solutions of Mie scattering by a soft sphere.
Therefore $\bb L$ represents the kinematic approximation to the solution
also known in quantum mechanics scattering theory as the first Born
approximation.

The cross-coupling matrix $\bb T$ accounts for multiple scattering effects.
If $\bb A$ is written $\bb A=\left(..a_{p;lm},b_{p;lm}...\right)^T$
$\square^T$ denoting transposition, then $\bb T$ is block diagonal :
%
\begin{equation*}
  \bb T =\left[
    \begin{array}{cccc}
      \bb 0      & ..~\bb T_{1q}~.. & ..~\bb T_{1p}~.. & \bb T_{1N}\\
      \bb T_{q1} & ..~\bb 0     ~.. & ..~\bb T_{qp}~.. & \bb T_{qN}\\
      \bb T_{p1} & ..~\bb T_{pq}~.. & ..~\bb 0     ~.. & \bb T_{pN}\\
      \bb T_{N1} & ..~\bb T_{Nq}~.. & ..~\bb T_{Np}~.. & \bb 0     \\
    \end{array}\right]
\end{equation*}
%
where $\bb T_{pq}$ represents the scattering from sphere $p$ due to the
scattering from sphere $q$. If the problem is fully coupled, the scattering
from sphere $p$ affects scattering from sphere $q$ and vice versa which
therefore requires inversion of the system.

As detailed further down below, backscattering can be neglected for very
fast electrons known as forward scattering approximation. This results
in $\bb T$ begin lower triangular if the spheres are sorted in ascending
order along $\bb e_z$. Inversion is therefore no longer necessary and
calculations can be performed sequentially.

\begin{figure}
  \centering
  \includegraphics[height=0.2\textheight]{figures/qdot2_approxCrop.png}
\caption{Two-level forward scattering approximation using the T-matrix approach.
Neglecting backscattering, scattering from the second scatterer located at
distance $kd$ from the first scatterer can be approximated by the sum of
1) the kinematic scattering to the incident beam and
2) the secondary scattering in response to the kinematic scattering from
the first scatterer. }\label{fig:qdot2_approx}
\end{figure}


Since $\bb A_0=\bb L$ represents single scattering, we can establish that
$\bb A_1=\bb T\bb A_0$ accounts for secondary scattering.
Similarly, outward scattering ampltiudes rom sphere $p$ can be written :
%
\begin{eqnarray}
  b_{p;lm} &=&
      b_{p;lm}^{(0)} + \sum_{q,z_q<z_p}T_{qp}b_{q;lm}^{(0)}  \label{eq:multiple_scattering}\\
    &&~+T_{pq}\sum_{q,z_q<z_p}T_{qr}\sum_{r,z_r<z_q}b_{r;lm}^{(0)} + ..\nonumber
\end{eqnarray}
%
where the first term accounts for kinematic scattering, the second term for
secondary scattering, the third term three time scattering and so on.
This is a similar development to the Korringa-Kohn-Rostoker (KKR) theory of
multiple scattering~\cite{Korringa1947,Kohn1954,Korringa1994}.
The forward multiple scattering approximation in the case of 2 scatterers
is illustrated in figure~\ref{fig:qdot2_approx}.





%%%%%%%%%%%%%%%%%%%%%%%%%%%%%%%%%%%%%%%%%%%%%%%%%%%%%%%%%%%%%%%%%%%%%%%%%
%%%%%%%%%%%%%%%%%%%%%%%%%%%%%%%%%%%%%%%%%%%%%%%%%%%%%%%%%%%%%%%%%%%%%%%%%
%%%% Multislice
%%%%%%%%%%%%%%%%%%%%%%%%%%%%%%%%%%%%%%%%%%%%%%%%%%%%%%%%%%%%%%%%%%%%%%%%%
%%%%%%%%%%%%%%%%%%%%%%%%%%%%%%%%%%%%%%%%%%%%%%%%%%%%%%%%%%%%%%%%%%%%%%%%%
\subsection{Multiple scattering in multislice}
In multislice, the forward scattering approximation is used and the
potential discretized in slices which are propagated from one slice to
the other using Fresnel propagator.
A multiple scattering interpretation by the multislice has been
proposed~\cite{ColweyModdie1957} which, although analgous to the
one presented above, differs in that it is stated in reciprocal space.
The expression for the scattering ampltiude $f(h,k)$ of beam $h,k$ is :
%
\begin{widetext}
% \begin{equation*}
\begin{eqnarray*}
  f(h,k) &\propto&
    \sum_{l}\sum_{h_1}\sum_{k_1}\sum_{l_1}..\sum_{h_{N-1}}\sum_{k_{N-1}}\sum_{l_{N-1}}
    Q_{h_1,k_1,l_1}..Q_{h_{N-1},k_{N-1},l_{N-1}}\\
  &&~~Q\Bigl(h-\sum_{n}^{N-1}h_n,k-\sum_{n}^{N-1}k_n,l-\sum_{n}^{N-1}l_n\Bigr)
  e^{-2\pi j\bigl(H\zeta - \Delta z\sum_{n=1}^{N-1}\zeta_n\bigr)}
% \end{equation*}
\end{eqnarray*}
\end{widetext}
%
where $h=\sum_{n=1}^N h_n$,$k=\sum_{n=1}^N k_n$,
$H=N\Delta z$ is the total thickness of the sample made of $N$ slices
of thickness $\Delta z$,
$Q_{h,k,l}=-j/\Delta z\delta_{h,k}e^{-2j\pi l_n z_n/c} + \sigma F_{h,k,l}$,
$F_{h,k,l}$ is the structure factor,
$\sigma=2\pi m_eh/\lambda$ the interaction parameter,
$\zeta$ the excitation error of beam $(h,l)$ and
$\zeta_n$ is the excitation error of beam
$\left(\sum_{r=1}^{n} h_r,\sum_{r=1}^n l_r\right)$,
and the excitation error is defined as :
%
\begin{equation*}
  \zeta = \frac{1}{2K}\Bigl(\frac{h}{a^2} + \frac{k}{b^2}\Bigr) - \frac{l}{c}
\end{equation*}
%
where $a$, $b$ and $c$ are the lattice constants of the crystal and
$K=1/\lambda$ the wave number.
This expression is the longitudinal distance of beam $(h,k,l)$ to
the Ewald paraboloid which is a very close to the Ewald sphere
for large wave vector. This is shown in figure~\ref{fig:MS_scattering}a.

% \lipsum
The From being scattered $n$ times
written only for beams in the zero order Laue zone (ZOLZ) is :


\begin{figure}
\begin{tabular}{c@{}c@{ }}
  \figsplit{0.23}{figs/readings/cowley/parabola.eps}&
  \figsplit{0.23}{figs/readings/cowley/scat3_1.eps}\\
  a) & b)
\end{tabular}
\caption{Multiple scattering in multislice.
a) Distance $\zeta^{(MS)}$ to the Ewald paraboloid(solid curve) as approximated by
multislice and distance $\zeta^{(BW)}$ to the Ewald sphere(dashed curve) known as the
excitation error in the blochwave theory. Point P is the projection of
reciprocal point $(u_n,w_n)$ onto the Ewald paraboloid.
b) Multiple scattering in reciprocal space for $N=3$ slices located at $z_1, z_2, z_3$.
The beam is scattered by $h_1,0$, then $h_2,0$ and then $h_N,0$ which results in an
overall contribution to reflection $h$.
The open blue circles show the subsequent excitation errors $\zeta^{(i)}$.
}\label{fig:MS_scattering}
\end{figure}








%%%%%%%%%%%%%%%%%%%%%%%%%%%%%%%%%%%%%%%%%%%%%%%%%%%%%%%%%%%%%%%%%%%%%%%%%
%%%%%%%%%%%%%%%%%%%%%%%%%%%%%%%%%%%%%%%%%%%%%%%%%%%%%%%%%%%%%%%%%%%%%%%%%
%%%% Numerical results
%%%%%%%%%%%%%%%%%%%%%%%%%%%%%%%%%%%%%%%%%%%%%%%%%%%%%%%%%%%%%%%%%%%%%%%%%
%%%%%%%%%%%%%%%%%%%%%%%%%%%%%%%%%%%%%%%%%%%%%%%%%%%%%%%%%%%%%%%%%%%%%%%%%
\section{Application and Results}

Although very efficient open source implementations are available for
electromagnetics~\cite{celes2017,pygmm2020}, an implementation suited for
solving Schr\"{o}dinger's equation has been made available~\cite{pyscat}.

\begin{figure}
\begin{tabular}{c@{}c@{ }}
  \figsplit{0.24}{figures/addth_error.png}&
  \figsplit{0.24}{figures/Tmatrix_cv.png}\\
  a) & b)
\end{tabular}
\caption{
a) Error between $\hl Y_{lm}$ for $l=4$ and $m=2$ computed at the origin and
using the translational addition theorem $\bb d_p = (0,3,5)$ with $\nu_{max}=10$.
The circle represent spheres of radius 1. Color axis in logscale.
b) Continuity error at the sphere boundaries in the T-matrix with increasing
order. For this example $N=4$ and $ka_{max}=4$.
}\label{fig:TmatrixError}
\end{figure}

In practice, the sums over the order $l$ has to be truncated to a integer.
A good rule to obtain accurate results is to take $l_{max}$ a few integer
above the maximum value of $ka$. Indeed, the spherical Bessel functions of
order $l$ have enough ripples to capture the periodicity of the incident wave
in the vicinity of the sphere. Moreover, the translational addition theorem
provides an approximation of the spherical Hankel functions with decreasing
accuracy from the center at which it is written similarly to a Taylor's expansion.
This is illustrated in Figure~\ref{fig:TmatrixError}a where the error between
$\hl Y_{lm}$ for $l=4$ and $m=2$ computed at the origin and using the
translational addition theorem with $\bb d_p = (0,3,5)$ and $\nu_{max}=10$ is
displayed in logscale.
This criteria for selecting $l_max$ is further confirmed by evaluating the
continuity of the wave functions at the spheres boundary as shown in
figure~\ref{fig:TmatrixError}b for $N=4$ and $ka_{max}=4$.
This can also be used to validate the correctness of the implementation
since it can be seen that machine accuracy can be reached with increasing the order.



%%%%%%%%%%%%%%%%%%%%%%%%%%%%%%%%%%%%%%%%%%%%%%%%%%%%%%%%%%%%%%%%%%%%%%%%%
%%%% Validity
%%%%%%%%%%%%%%%%%%%%%%%%%%%%%%%%%%%%%%%%%%%%%%%%%%%%%%%%%%%%%%%%%%%%%%%%%
\subsection{Validity of forward scattering and phase grating approximation for light atoms}

\begin{figure}
  % \centering
  \hspace{-1em}
  \begin{tabular}{c@{}c@{ }}
    \figsplit{0.24}{figures/qdotSphereSingle_shells_pot1.eps}&
    \figsplit{0.24}{figures/qdotSphereSingle_shells_fka1.eps}\\
    a) & b)
  \end{tabular}
\caption{
a) Electrostatic potential created by an atom of carbone. The dashed line
shows the Coulomb potential usually used in the IAM using fitted
parameteres from \cite{Kirkland2019}. The red solid line shows the best
fit using a single screened coulomb potential whereas the blue solid line
the  best fit using a regular Coulomb potential with offset.
The blue patches show a multi-shell approximation model which can be used
to represent the potential using a T-matrix approach.
b) Scattering amplitude in the Born approximation for the multi-shell model
with increasing truncation radius. Although the potential is very small
at large radius, a value of $ka_{max}=325$ is necessary to account properly
for the angular spread of the scattering.
}\label{fig:IAM}
\end{figure}

In the case of very fast electrons typically used in transmission electron
microscopes $E=50-300keV$. Inclusion of relativistic effects result in
$\lambda=0.025\AA@200keV$ which will be assumed from now on unless stated
otherwise.
In the IAM, the potential is the sum of the Coulomb potential created
by the charge of the nucleus and the electron cloud. It can be approximated
with a sum of screened Coulomb potential and Gaussian terms ~\cite{Kirkland2019}.
For typical light atoms such as commonly found in organic compounds, a single
screend Coulomb potential term can be a pretty good approximation as shown in
figure~\ref{fig:IAM}a. The solution to Schrodinger's equation in such a
potential can only be solved perturbatively~\cite{Muller1965} and is beyond the
scope of this document. However, orders of magnitude for $k_p$ and $ka_p$
can be estimated by a multi-shell representation as shown in figure~\ref{fig:IAM}a.
Although the range of the potential is theoretically infinite, it can
reliably be truncated to radius $ka$ by using the Born approximation.
In figure~\ref{fig:IAM}b the multi-shell scattering amplitudes are shown for
increasing values of truncation radius. A satisfactory agreement with the
electron diffraction scattering factors is obtained for normalized raidus
as large as $ka=350$ to account for the proper low angle representation
although the potential is very small at such a radius.
As it was seen above, such a large radius would be quite hard to simulate with
the T-matrix due to the large orders to be included. However, it can be
argued that in the interest of the multiple scattering interpretation
presented in this paper, appreciable multiple scattering


%%%%%%%%%%%%%%%%%%%%%%%%%%%%%%%%%%%%%%%%%%%%%%%%%%%%%%%%%%%%%%%%%%%%%%%%%
%%%% Application of Multiple scattering
%%%%%%%%%%%%%%%%%%%%%%%%%%%%%%%%%%%%%%%%%%%%%%%%%%%%%%%%%%%%%%%%%%%%%%%%%
\subsection{Multiple scattering approximations}

\begin{figure}
\begin{tabular}{c@{}c@{ }}
  \figsplit{0.24}{figures/qdotSphereArray1_log_ska.pdf}&
  \figsplit{0.24}{figures/qdotSphereArray1_n0_fka.pdf}\\
  a) & b)
\end{tabular}
\caption{
a) Normalized scattering cross section in $\sigma$ in $\log$for a few values
of $k_p$ over a range of normalized radius $ka$.
b) Scattering amplitude for a few normalized radius $ka$ and potential
strength $k_p$.
}\label{fig:TmatrixSingle}
\end{figure}


It is noted that in the weak scattering regime $kp-1\ll 1$ and not too large spheres $ka$, the shape of the diffraction pattern is identical to the Born approximation. Only the amplitude of the scattering cross section increases with radius.

\begin{figure}
  \centering
    % \begin{subfigure}
    \includegraphics[height=0.35\textheight]{figures/qdotSphereArray2approx_kakp_forward.png}
  \caption{$(ka,k_p)$ map(color axis in logscale) of the error of $b_{p;lm}$
using the forward scattering approximation.
The blue dot correspond to the location of the spherical shells of Carbone
atom at $E=200keV$.}\label{fig:TmatrixApproxForward}
\end{figure}
  % \end{subfigure}


Both the uncoupled and forward scattering approximation work better with increasing distances $kd$ since scattering from the spheres reduces with distance. It is therefore less likely to affect scattering from the other spheres.
The low values of $k_p$ result in overall good approximation of both the uncoupled and forward scattering approximation. This is an anticipated result since for weak potentials, the kinematic approximation is more valid.
The uncoupled approximation improves with small radii since Small $ka$ result in small scattering cross section,
On the other hand the forward scattering approximation improves with larger values of $ka$ since backward scattering is less likely for large $ka$.

\begin{figure}
  \centering
    \includegraphics[height=0.35\textheight]{figures/qdotSphereArrayNapprox_err_kp2.pdf}
  \caption{$\log_{10}err(b_{p;lm})$ the scattering amplitudes coefficients
  with increasing number of spheres for a few normalised radius $ka$
  using $k_p=1.01$.
}\label{fig:TmatrixApproxErr}
\end{figure}


\begin{figure}
  \centering
    \includegraphics[height=0.35\textheight]{figures/qdot_vs_multi.eps}
  \caption{Comparison of multislice and Tmatrix for increasing number of
  spheres $N$. $keV=50$, $ka=11$, $kd=3ka$, $k_p=1.001$.
}\label{fig:MSvsTmatrix}
\end{figure}

% \begin{figure}
% \begin{tabular}{c@{}c@{ }}
%   \figsplit{0.24}{figures/qdotSphereArray2approx_kakp_forward.png}&
%   \figsplit{0.24}{figures/qdotSphereArrayNapprox_err_kp2.pdf}\\
%   a) & b)
% \end{tabular}
% \caption{
% a)
% b)
% }\label{fig:TmatrixApprox}
% \end{figure}




%%%%%%%%%%%%%%%%%%%%%%%%%%%%%%%%%%%%%%%%%%%%%%%%%%%%%%%%%%%%%%%%%%%%%%%%%
%%%%%%%%%%%%%%%%%%%%%%%%%%%%%%%%%%%%%%%%%%%%%%%%%%%%%%%%%%%%%%%%%%%%%%%%%
%%%% Conclusion and Perspective
%%%%%%%%%%%%%%%%%%%%%%%%%%%%%%%%%%%%%%%%%%%%%%%%%%%%%%%%%%%%%%%%%%%%%%%%%
%%%%%%%%%%%%%%%%%%%%%%%%%%%%%%%%%%%%%%%%%%%%%%%%%%%%%%%%%%%%%%%%%%%%%%%%%
\section{Conclusion}

An alternative approach based on the T-matrix has been applied to the
scattering of fast electrons by light-atom structures.
The validity of important approximations used in multislice has been
discussed and a multiple scattering approximation framework has been proposed
and compared to other existing interpretations.

Although the spherically symmetric effective potential does not accurately
model the potential used in atoms, it was shown that the multiple scattering
interpretation should equally apply to the more accurate case of a screened
Coulomb potential. A possible inclusion of such a potential could be performed
by using a family of basis functions consistent with the exisiting variational
based solutions of the Schrodinger's equation in a Yukawa potential.
The main ultimate limitation of both this approach and the traditional
multislice lies in the use of the independent atom model which by definition
ignores the effect of bonding which may be relevant for structure determination
of organic structures. However, it is still an open question whether such bonding
play an important role in HEED.

The advantage of a multiple scattering approximation approach is that it offers
both the possibility of a massively parallel computation of dynamical diffraction
while including incoherent inelastic scattering with a stochastic approach.
It is indeed strongly anticipated that inelastic scattering has a dramatic
mitigation effect of dynamical diffraction even when energy filters are used.
