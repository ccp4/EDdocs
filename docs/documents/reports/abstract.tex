% \documentclass[a4paper,10pt]{article}
% \newcommand{\path}{../../../../mygit/_Latex/}
% \input{\path StdPack.tex}
% \usepackage{\path cmds}
% \usepackage{url}
% \newcommand{\bibTitle}[1]{``#1''}
% \graphicspath{{figures/}}
% \begin{document}

\section{Introduction}
Macromolecular structures have been successfully solved with
Electron diffraction(ED) patterns and standard macromolecular X-ray crystallographic(MX) techniques since 2013~\cite{ShiNanenga2016,ClabbersGrueneAbrahams2017}.
This technique is currently refered to as microED.
In practice, growing good quality macromolecular crystals up to micrometric sizes is often a challenge and even sometimes impossible.
MicroED is therefore a very appealing technique because it enables solving structures from nanocrystals. Another interesting aspect is that ED patterns provide information about the electrostatic potential which is a complementary information to the electron density maps provided by X-ray diffraction patterns. Besides, ED patterns may provide higher resolution than the more popular cryo-EM imaging technique~\cite{Latychevskaia2019}.
However, theoretical works~\cite{GlaeserDowning1993,SubramanianSpence2015} have suggested that dynamical diffraction effects are too prominent for macromolecular crystals larger than a few tens of nanometer to allow the use of standard MX techniques for structure determination.

In this work, simulations of ED patterns are performed with the multislice algorithm(MS)~\cite{CowleyMoodie1957,Ishizuka2004,Kirkland2019} as an
attempt to explain the discrepancies between theory and experiment.

The first section presents the MS algorithm, the second section shows an example for a 2-beam diffraction setup
The third section presents simulates are performed on small molecules such as biotin.


% %%%%%%%%%%%%%%%%%%%%%%%%%%%%%%%%%%%%%%%%%%%%%%%%%%%%%%%%%%%%%%%%%%%%%%%
% % End Document
% %%%%%%%%%%%%%%%%%%%%%%%%%%%%%%%%%%%%%%%%%%%%%%%%%%%%%%%%%%%%%%%%%%%%%%%
% \newpage
% \bibliography{\path library}
% \bibliographystyle{ieeetr}
% 	% \bibitem{Holton}
% 	% 	J. Holton, \em{Near Bragg algorithm},
%   %   \url{https://bl831.als.lbl.gov/~jamesh/nearBragg/};
% \addcontentsline{toc}{section}{References}
% \end{document}
