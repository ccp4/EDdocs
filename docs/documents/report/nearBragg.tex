\documentclass[a4paper,10pt]{article}
\newcommand{\path}{../../../../mygit/_Latex/}
\input{\path StdPack.tex}
\usepackage{\path cmds}
\usepackage{url}
\newcommand{\bibTitle}[1]{``#1''}
\graphicspath{{figures/}}
\begin{document}



\section{Near Bragg : Hydrid particle-wave approach}
An alternative approach to the multislice algorithm consists in solving~\eqref{eq:FESE} approximately using an extension of the kinematic theory to multiple scattering. This method is called near-bragg(NB) and initially developed for X-rays~\cite{Holton}.

Each atom scatter the incident electron beam according to its electron-atom scattering cross section. The path length of the incident beam is computed and interference at computed at selected pixels on a detector some distance away from the sample.

\subsection{Scattering by individual atoms}
The first order approximation yields the kinematic theory of diffraction also known as first Born approximation. It is a perturbation treatment often used to allow for an analytical treatment. The incident state $\phi$ is used in place of $\Psi$ in the right hand side so that the scattering equation can readily be solved as :
\begin{equation}
    \Psi(\bb r)=e^{ikz} + f(\theta)\frac{e^{i\bb k\cdot\bb r}}{|\bb r|}
\end{equation}
where :
\begin{equation}
    f(\theta) = -\frac{2me}{h^2}\int d^3re^{i\bb q\cdot\bb r}V(r)
~~\mbox{,     }~~
    \frac{d\sigma}{d\Omega} = |f(\theta)|^2
\end{equation}

where $f(\theta)$ is the scattering amplitude, i.e. the Fourier transform of the electrostatic potential.
In the first Born approximation the far field diffraction pattern is proportional to the square of the scattering amplitude which is known as the differential cross section $\sigma$.


\subsection{Central beam calculation}
With the MS method, the intensity of the central beam corresponds to the sum of the forward scattering beam and the incident coherent unscattered beam.
On the other hand, NB is based on path length calculation to compute the scattered beams contribution to the diffraction pattern.
The central unscattered beam and the central forward scattered beam are therefore computed separetely.
Below, the scattering cross section is used to determine the amount of unscattered beam and will be subsequently used for proper bookeeping in the implementation of double scattering.

\subsubsection{Cross section, mean free path and scattering probabilities}
In a continuous medium, the probability of an electron to undergo $m$ elastic collisions and $n$ inelastic collisions after going through a specimen of length $z$ follows the Poisson distribution~\cite{Latychevskaia2019}:
\begin{equation}\label{eq:PoissonProba}
  P_{mn}(z) =
    \frac{1}{m!}\left(\frac{z}{l_e}\right)^me^{-z/l_e}
    \frac{1}{n!}\left(\frac{z}{l_i}\right)^ne^{-z/l_i}
\end{equation}
where $l_e=1/\sigma_e\rho$ is the average elastic collision mean free path $\sigma_e=|f^{(e)}_a|^2$ being the interaction cross section and $f^{(e)}_a$ the atomic elastic scattering factor, $l_i$ the average inelastic collision mean free path and $\rho$ is the number of atoms per unit volume (per unit area in 2D).

For a typical protein, $\rho=106$ atoms per $nm^3$, the average scattering cross section $\sigma_e=0.001-0.005A^2$ (within beam energy range $E=100-1000keV$)  resulting in mean free path on the order of $l_e=200-1000nm$. The corresponding probabilities~\eqref{eq:PoissonProba}are shown figure~\ref{fig:Pcoh_kin_dual_dyn}. It is apparent that for the low

  \begin{figure}[h!]
  	\begin{subfigure}{0.32\textwidth}
  		\centering
      \def\svgwidth{\columnwidth}
  		\input{figures/Pcoh_kin_dual_dyn0.pdf_tex}
  		\caption{}\label{fig:Pcoh_kin_dual_dyn0}
  	\end{subfigure}
  	\begin{subfigure}{0.32\textwidth}
  		\centering
      \def\svgwidth{\columnwidth}
  		\input{figures/Pcoh_kin_dual_dyn1.pdf_tex}
  		\caption{}\label{fig:Pcoh_kin_dual_dyn1}
  	\end{subfigure}
    \begin{subfigure}{0.32\textwidth}
  		\centering
      \def\svgwidth{\columnwidth}
  		\input{figures/Pcoh_kin_dual_dyn2.pdf_tex}
  		\caption{}\label{fig:Pcoh_kin_dual_dyn2}
  	\end{subfigure}

  	\caption[protein multiple scattering and mean free path]{
    Probability of no scattering(blue), single scattering, i.e. kinematic scattering(green), double scattering(orange) and multiple scattering(solid black) as a function of crystal thickness for a typical protein crystal with average cross section and mean free path
  		\ref{fig:Pcoh_kin_dual_dyn0} $\sigma_e=0.001\ang^2$,$l_e=943nm$,
  		\ref{fig:Pcoh_kin_dual_dyn1} $\sigma_e=0.003\ang^2$,$l_e=377nm$,
  		\ref{fig:Pcoh_kin_dual_dyn2} $\sigma_e=0.005\ang^2$,$l_e=188nm$.
  	}\label{fig:Pcoh_kin_dual_dyn}
  \end{figure}


\subsubsection{Application in Near Bragg}

Considering a primitive orthorombic lattice with parameters $a_x$ $b_z$, a single atom per unit cell, with cross section $\sigma_e$ the probability of the unscattered beam is found at cell $N+1$ from cell $N$ as :
\begin{equation}\label{eq:centralProbaRect}
  P^{(coh)}_{N+1} =
    P^{(coh)}_{N}\left(1-\frac{dz}{l_e}\right) =
    P^{(coh)}_{N}\left(1-\frac{\sigma_e}{a_x}\right)
\end{equation}
where $l_e=a_xb_z/\sigma_e$ and $dz=b_z$. The solution of ~\eqref{eq:centralProbaRect} when $dz\rightarrow 0$ results in ~\eqref{eq:PoissonProba} for $m=0$.

In MS, the intensity $I_0$ is normalized so it integrates to unity over the transverse plane and therefore represents the density of probability of the electron.
For a structure with a basis and $N_x$ transverse unit cells, the atoms are arranged in slices of $n_k$ atoms.
The incident plane wave intensity is therefore $I_0=1/N_xa_x$ and the probability of an electron of being unscattered is originally $P^{(coh)}_0=1$ and the other scattering probabilities $P^{(scat)}_0=0$.

The fraction of electrons scattered per unit time by the array of atoms in slice $k$ is
$P^{(scat)}_k=\sum_{j=1}^{n_k}\sigma_e(j)I_0$.
The number of unscattered electrons per unit time left after going through the slice $k$ is $P^{(coh)}_{k+1}=P^{(coh)}_k-P^{(scat)}_k$
and the unscattered intensity after going through the slice is
$I_0=P^{(coh)}_{k+1}/N_xa_x$.
The evolution of the probability of an electron for being unscattered at slice $k+1$ after going through slice $k$ is therefore :
\begin{equation}\label{eq:centralProba}
  P^{(coh)}_{k+1} = P^{(coh)}_{k}\left(1-\sum_{j=1}^{N_k}\frac{\sigma_e(j)}{N_xa_x}\right)
\end{equation}



\subsection{Kinematic calculation}
In the kinematic approximation, every atom contributes once (single scattering) to the interference pattern at the detector.
More precisely, the contribution of atom $j$ to the intensity value at detector pixel $i$ is given by the interference term $\exp(ik_0 R_{ij})/R_{ij}$ where  $R_{ij}$ is the path length from atom $j$ to pixel $i$.
It is written with increasing level of approximations as :
\begin{eqnarray}
R_{ij}
      &\underset{Greens}{=}& \sqrt{\left(x-x_0\right)^2+\left(z-z_0\right)^2} \\
       &\underset{Fresnel}{\approx}&     \left(z_0-z\right) + \frac{\left(x-x_0\right)^2}{2\left(z_0-z\right)} \\
       &\underset{Fraunhofer}{\approx}&  \left(z_0-z\right) + \frac{x_0^2}{2z_0} - \frac{xx_0}{z_0}  \\
\end{eqnarray}
where $x_0,z_0$ are the pixel positions and $x,z$ are the atom position.
For planar illumination, the path length from the source the to atom $j$ is $z$ and must be added to the path length.

The angle of scattering is determined with $\sin(\theta)=|x-x_0|/|z-z_0|$ so the atomic form factor $f(k_0\sin(\theta))$ is used to apply the correct contribution to the scattering amplitude at that pixel.


The validity of the kinematic approximation is established by comparing NB to MS with a simple square structure with lattice constants $a_x=10,b_z=5$ with $100\times100$ unit cells.
The strenght of the potential is varied to observe the onset of dynamical diffraction.

Figure~\ref{fig:MSvsNBweak} shows a comparison between MS and NB for a weak potential $\eps=0.001$. In figure~\ref{fig:MSvsNBweak_I}, the agreement for a weak potential between the NB and MS diffraction patterns after $50nm$ propagation is good.

\begin{figure}[h!]
	\begin{subfigure}{0.24\textwidth}
		\centering
    \def\svgwidth{\columnwidth}
		\input{figures/MSvsNBweak_V.pdf_tex}
		\caption{}\label{fig:MSvsNBweak_V}
	\end{subfigure}
	\begin{subfigure}{0.24\textwidth}
		\centering
    \def\svgwidth{\columnwidth}
		\input{figures/MSvsNBweak_Iz.pdf_tex}
		\caption{}\label{fig:MSvsNBweak_Iz}
	\end{subfigure}
  \begin{subfigure}{0.24\textwidth}
		\centering
    \def\svgwidth{\columnwidth}
		\input{figures/MSvsNBweak_Bz.pdf_tex}
		\caption{}\label{fig:MSvsNBweak_Bz}
	\end{subfigure}
	\begin{subfigure}{0.24\textwidth}
		\centering
    \def\svgwidth{\columnwidth}
		\input{figures/MSvsNBweak_I.pdf_tex}
		\caption{}\label{fig:MSvsNBweak_I}
	\end{subfigure}

	\caption[MS vs NB weak]{
		\ref{fig:MSvsNBweak_V} weak potential strength $\eps=0.001$.
		\ref{fig:MSvsNBweak_Iz} Evolution of diffraction pattern as function of thickness.
		\ref{fig:MSvsNBweak_Bz} Evolution of major beam intensities as function of thickness.
		\ref{fig:MSvsNBweak_I} comparison of diffraction patterns $t=50nm$ for NB and MS.
	}\label{fig:MSvsNBweak}
\end{figure}

Figure~\ref{fig:MSvsNBstrong} shows on the other hand, that for a stronger potential strength $\eps=0.5$, the NB kinematic approximation is not acceptable after $50nm$ propagation.

\begin{figure}[h!]
	\begin{subfigure}{0.24\textwidth}
		\centering
    \def\svgwidth{\columnwidth}
		\input{figures/MSvsNBstrong_V.pdf_tex}
		\caption{}\label{fig:MSvsNBstrong_V}
	\end{subfigure}
	\begin{subfigure}{0.24\textwidth}
		\centering
    \def\svgwidth{\columnwidth}
		\input{figures/MSvsNBstrong_Iz.pdf_tex}
		\caption{}\label{fig:MSvsNBstrong_Iz}
	\end{subfigure}
  \begin{subfigure}{0.24\textwidth}
		\centering
    \def\svgwidth{\columnwidth}
		\input{figures/MSvsNBstrong_Bz.pdf_tex}
		\caption{}\label{fig:MSvsNBstrong_Bz}
	\end{subfigure}
	\begin{subfigure}{0.24\textwidth}
		\centering
    \def\svgwidth{\columnwidth}
		\input{figures/MSvsNBstrong_I.pdf_tex}
		\caption{}\label{fig:MSvsNBstrong_I}
	\end{subfigure}

	\caption[MS vs NB strong]{
		\ref{fig:MSvsNBstrong_V} strong potential strength $\eps=0.5$.
		\ref{fig:MSvsNBstrong_Iz} Evolution of diffraction pattern as function of thickness.
		\ref{fig:MSvsNBstrong_Bz} Evolution of major beam intensities as function of thickness.
		\ref{fig:MSvsNBstrong_I} comparison of diffraction patterns $t=50nm$ for NB and MS.
	}\label{fig:MSvsNBstrong}
\end{figure}



\subsection{Double scattering calculation}



\begin{figure}[h!]
	\begin{subfigure}{0.33\textwidth}
		\centering
    \def\svgwidth{\columnwidth}
		\input{figures/NBproba.pdf_tex}
		\caption{}\label{fig:NBproba}
	\end{subfigure}
	\begin{subfigure}{0.33\textwidth}
		\centering
    \def\svgwidth{\columnwidth}
		\input{figures/NBsingle.pdf_tex}
		\caption{}\label{fig:NBsingle}
	\end{subfigure}
  \begin{subfigure}{0.33\textwidth}
		\centering
    \def\svgwidth{\columnwidth}
		\input{figures/NBdouble.pdf_tex}
		\caption{}\label{fig:NBdouble}
	\end{subfigure}

	\caption[Near Bragg scattering probabilities]{
  Scattering probabilities in the Near Bragg implementation
		\ref{fig:NBproba} global,
		\ref{fig:NBsingle} single scattering distribution  every 100 slices,
    \ref{fig:NBdouble} same as~\ref{fig:NBsingle} double scattering.
	}\label{fig:NBprobabilities}
\end{figure}





%%%%%%%%%%%%%%%%%%%%%%%%%%%%%%%%%%%%%%%%%%%%%%%%%%%%%%%%%%%%%%%%%%%%%%%
% End Document
%%%%%%%%%%%%%%%%%%%%%%%%%%%%%%%%%%%%%%%%%%%%%%%%%%%%%%%%%%%%%%%%%%%%%%%
\newpage
\bibliography{\path library}
\bibliographystyle{ieeetr}
	% \bibitem{Holton}
	% 	J. Holton, \em{Near Bragg algorithm},
  %   \url{https://bl831.als.lbl.gov/~jamesh/nearBragg/};
\addcontentsline{toc}{section}{References}
\end{document}
